\documentclass[12pt,oneside]{book}

\usepackage{listings}
\usepackage{xcolor}
\usepackage{tcolorbox}

% Color Definitions
\definecolor{codegreen}{rgb}{0,0.6,0}
\definecolor{codegray}{rgb}{0.5,0.5,0.5}
\definecolor{codepurple}{rgb}{0.58,0,0.82}
\definecolor{backcolour}{rgb}{0.95,0.95,0.92}

% Listing - TypeScript (Lang Definition)
\lstdefinelanguage{TypeScript}{
    keywords={ 
        export, interface, extends, keyof, 
        function, number, string, readonly,
        void },
    sensitive=true,
    comment=[l]{//},
    comment=[s]{/*}{*/},
    string=[b]",
}

\lstalias{ts}{TypeScript}
\lstalias{typescript}{TypeScript}

% Listing - Style
\lstdefinestyle{mystyle}{
    backgroundcolor=\color{backcolour},   
    commentstyle=\color{codegreen},
    keywordstyle=\color{magenta},
    numberstyle=\tiny\color{codegray},
    stringstyle=\color{codepurple},
    basicstyle=\ttfamily\footnotesize,
    breakatwhitespace=false,         
    breaklines=true,                 
    captionpos=b,                    
    keepspaces=true,                 
    numbers=left,                    
    numbersep=5pt,                  
    showspaces=false,                
    showstringspaces=false,
    showtabs=false,                  
    tabsize=2
}

\newcommand{\code}[1]{
    \lstset{ basicstyle=\ttfamily } % Set Inline Style
    \colorbox{codegray}{\lstinline|#1|}
    \lstset{ style=mystyle } % Set General Style
}

\title{{\Large Bot Specification Sheet}\\{\small A guide to `pp-cli'}}
\author{dkantereivin, Doomer}
\date{\today}

\begin{document}
\maketitle
\tableofcontents
\chapter{Introduction}
    This is the Bot Specification - here every feature and their usage will
    be documented! In the first chapter the core structure of the bot is 
    explained in detail and some decisions concerning the project organization
    are discussed. Apart from general design decisions, Chapter 1 will also 
    include a quick overview over the project goals and milestones for the 
    near future.

\chapter[Core]{Core Structure}
    `pp-cli' consists out of 3 components, which are crucial for it to work!
    Events Handlers, Commands and Services are the three backbones of this 
    project.\\
    Event Handlers take Events from the Discord API and work with them - for 
    example, fetch them to detect and eventually trigger commands or use them 
    to classify data for a future machine learning subproject!\\
    Commands are the part of the project that is triggered by certain events
    and usually processes and outputs a user message or accesses and possibly
    alters the database (soon to be implemented)!\\
    Services - the final piece of that triplet is responsible for providing 
    universal, reusable code snippets and are categorized into specific 
    areas of application, some of them are needed for the bot to start up, 
    while others take care of the functions during the bots runtime!

\chapter[Events]{Event Handling}
    Let's start with the first concept - the Event Handlers. The Event Handlers
    are quite simply organized structure-wise. The interface used as a blueprint
    for the event handlers contains a single field - \code{EVENT_NAME} and a 
    single \code{onEvent} function, which automatically adapts to the arguments
    required for the specified Event Name such as seen in Listing 
    \ref{lst:event-interface}!\\
    \begin{minipage}{\linewidth}
        \begin{lstlisting}[language=TypeScript, caption=Event Handler Interface, 
        label=lst:event-interface]
interface IEventHandler<EventName extends keyof ClientEvents> 
{
    readonly EVENT_NAME: EventName;
    onEvent(...args: ClientEvents[EventName]): void;
}
        \end{lstlisting}
    \end{minipage}
    At the moment we implemented 3 such event handlers. Ready Handler, for which
    the event gets fired at the bot start up. Message Handler, for which the 
    event fires when the bot receives a message and a Message Reaction Add Handler,
    which handles incoming reactions for cached messages! All those play an important
    part for the application as a whole, and enable certain feature sets.\\
    The Ready Handler prepares the Bot for action - this will be used in the near 
    future to initialize the caches. The current implementation is as easy as the
    following:\\
    \begin{minipage}{\linewidth}
        \begin{lstlisting}[language=TypeScript, caption=Ready Handler, label=lst:ready-handler]
public readonly EVENT_NAME = 'ready';
public onEvent() {
    Logger.info('Connected.');
}
        \end{lstlisting} 
    \end{minipage}
    The Message Handler is responsible for 
    testing whether a command triggers given the message supplied by the event.
    The Reaction Handler tests for certain tags and if those apply it flags the message
    and adds it to the database (to be implemented)!


\chapter[Commands]{Commands \& Triggers}

\chapter[Services]{Services}

\end{document}